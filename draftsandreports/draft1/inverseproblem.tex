\section{The Inverse Problem}
In this section we formulate the inverse problem of estimating 
the source emission rates based on measurements of the 
concentration. We present models for measuring the different 
kinds of data that are available and also formulate the problem 
as a multi-objective optimization problem. Details of the
optimization algorithm and validation are presented in the 
next section. \\

The Gaussian plume model that was discussed in section 2 provides 
us with approximation of the pollutant concentration at each 
time step. But this is not the value that is measured by the 
instrument and an appropriate model for the measurements is crucial in 
solution of the inverse problem. In this project we are interested in 
instruments which provide measurements that are related to accumulated 
concentrations of the pollutant.